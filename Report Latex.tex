
\documentclass{article}
\usepackage{graphicx}
\usepackage{float} % To use figures with H option
\usepackage{hyperref} % For adding hyperlinks

\title{React Native Development Environment Setup and Todo List Application}
\author{Harshil Sharma}
\date{November 17, 2024}

\begin{document}

\maketitle

\begin{center}
\href{https://github.com/xxender13/Lab3}{GitHub Repository: https://github.com/xxender13/Lab3}
\end{center}

\section{Introduction to React Native}
React Native is an open-source framework developed by Facebook that allows developers to create mobile applications using JavaScript and React. Its primary advantage is the ability to develop apps for both iOS and Android platforms from a single codebase, significantly reducing development time and effort. This means you can write your application once and deploy it on both platforms, taking advantage of native performance and user experience.

\subsection{Key Features of React Native}
\begin{itemize}
    \item \textbf{Cross-Platform Compatibility:} Build applications that run seamlessly on both iOS and Android.
    \item \textbf{Hot Reloading:} Instantly see the results of changes made to your code, speeding up the development process.
    \item \textbf{Rich Ecosystem:} Access a wide range of libraries and components, making it easier to add complex functionalities.
\end{itemize}

% After your introduction section but before the subsections

\noindent\textbf{\Large Task 1: Set Up the Development Environment (50 Points)}\\[0.5cm]
\noindent In this task, I set up the development environment to build React Native applications.\\[1cm]

\subsection{Step 1: Install Node.js and Watchman}
To complete this step:
\begin{enumerate}
    \item Installed Node.js from the \href{https://nodejs.org/en}{official website}, which also included npm.
    \item Installed Watchman (optional) using Homebrew on macOS:
    \begin{verbatim}
    brew install watchman
    \end{verbatim}
\end{enumerate}

\subsection{Step 2: Install React Native CLI}
Installed the React Native CLI using:
\begin{verbatim}
    npm install -g react-native-cli
\end{verbatim}
Alternatively, used npx:
\begin{verbatim}
    npx react-native init YourProjectName
\end{verbatim}

\subsection{Step 3: Set Up Android Studio (or Xcode for iOS)}
For Android:
\begin{enumerate}
    \item Installed Android Studio and enabled SDK tools including Android SDK Build-Tools, Platform-Tools, Emulator, and Google Play Services.
    \item The setup was verified as shown in Figure~\ref{fig:sdk_setup}.
\end{enumerate}
For iOS:
\begin{itemize}
    \item Installed Xcode and command line tools using:
    \begin{verbatim}
    xcode-select --install
    \end{verbatim}
\end{itemize}

\begin{figure}[H]
    \centering
    \includegraphics[width=0.7\textwidth]{SDK Setup.jpg}
    \caption{Android SDK Setup in Android Studio}
    \label{fig:sdk_setup}
\end{figure}

\subsection{Step 4: Create a New React Native Project}
Initialized a new project using:
\begin{verbatim}
    npx react-native init YourProjectName
    cd YourProjectName
\end{verbatim}

\subsection{Step 5: Open the Project in Visual Studio Code}
Opened the folder in VS Code and installed the React Native Tools extension.

\subsection{Step 6: Start the Metro Bundler}
Started the Metro Bundler using:
\begin{verbatim}
    npx react-native start
\end{verbatim}

\begin{figure}[H]
    \centering
    \includegraphics[width=0.7\textwidth]{Startmetrobumdler.jpg}
    \caption{Metro Bundler Started in Terminal}
    \label{fig:metro_bundler}
\end{figure}

\subsection{Step 7: Run the App on Emulator or Device}
For Android:
\begin{verbatim}
    npx react-native run-android
\end{verbatim}
For iOS:
\begin{verbatim}
    npx react-native run-ios
\end{verbatim}

\begin{figure}[H]
    \centering
    \includegraphics[width=0.5\textwidth]{ApponEmulator.jpg}
    \caption{App running on Android Emulator}
    \label{fig:emulator}
\end{figure}

\subsection{Step 8: Run the App Using Expo}
Installed and created a new Expo project:
\begin{verbatim}
    npm install -g expo-cli
    npx expo init YourProjectName
    npx expo start
\end{verbatim}
Connected a physical device using the Expo Go app.

\begin{figure}[H]
    \centering
    \includegraphics[width=0.5\textwidth]{Physical Device.jpeg}
    \caption{App running on Physical Device using Expo}
    \label{fig:expo_device}
\end{figure}

\subsection{Submission Requirements for Task 1}
\begin{itemize}
    \item \textbf{Screenshots}: 
    \begin{itemize}
        \item Figure~\ref{fig:emulator} shows the app running on the Android emulator.
        \item Figure~\ref{fig:expo_device} shows the app running on a physical device using Expo.
        \item Figure~\ref{fig:metro_bundler} shows the Metro Bundler running in the terminal.
    \end{itemize}
    \item \textbf{Setting Up an Emulator}: Steps to set up the emulator are explained in Section~\ref{fig:sdk_setup}. Challenges faced included issues with hardware acceleration, which were resolved by enabling virtualization in BIOS.
    \item \textbf{Running on a Physical Device Using Expo}: The process for running the app on a physical device is explained in Step 8, including troubleshooting connection issues.
    \item \textbf{Comparison of Emulator vs. Physical Device}: 
    \begin{itemize}
        \item Advantages of Emulator: Easier debugging, supports multiple device configurations.
        \item Disadvantages of Emulator: Requires more resources, can be slower.
        \item Advantages of Physical Device: Realistic testing experience, better performance.
        \item Disadvantages of Physical Device: Requires manual setup, debugging can be cumbersome.
    \end{itemize}
    \item \textbf{Troubleshooting Common Errors}: 
    \begin{itemize}
        \item Encountered an issue with the default \texttt{App.tsx} file. Resolved by changing the extension to \texttt{App.js}.
        \item JAVA\_HOME path was not being verified. Used an LLM to obtain the correct command for Visual Studio to fix the path.
    \end{itemize}
\end{itemize}

\section{Task 2: Building a Simple To-Do List App (60 Points)}
In this task, I built a simple To-Do List application using React Native.

\subsection{App Features}
\begin{itemize}
    \item \textbf{Add New Tasks:} Users can input text into a form and add it as a task to the to-do list.
    \begin{figure}[H]
        \centering
        \includegraphics[width=0.5\textwidth]{task on emu.jpg}
        \caption{Adding Task on Emulator}
    \end{figure}
    \begin{figure}[H]
        \centering
        \includegraphics[width=0.5\textwidth]{toggle on pd.jpg}
        \caption{Adding Task on Physical Device}
    \end{figure}

    \item \textbf{Update Existing Tasks:} Users can modify tasks they have already created.
    \begin{figure}[H]
        \centering
        \includegraphics[width=0.5\textwidth]{updated task on emu.jpg}
        \caption{Editing Task on Emulator}
    \end{figure}
    \begin{figure}[H]
        \centering
        \includegraphics[width=0.5\textwidth]{updating taskon pd.jpg}
        \caption{Editing Task on Physical Device}
    \end{figure}
    
    \item \textbf{Delete Tasks:} Users can remove tasks from the list.
    \begin{figure}[H]
        \centering
        \includegraphics[width=0.5\textwidth]{deleteonemu.jpg}
        \caption{Deleting Task on Emulator}
    \end{figure}
    \begin{figure}[H]
        \centering
        \includegraphics[width=0.5\textwidth]{Delete on PD.jpeg}
        \caption{Deleting Task on Physical Device}
    \end{figure}
    
    \item \textbf{Scrollable Task List:} The to-do list supports scrolling, allowing navigation through a large number of tasks.

    \item \textbf{User-Friendly Interface:} The app provides a simple and intuitive interface for managing tasks.
\end{itemize}

\subsection{Step 1: Set Up the Project}
\begin{enumerate}
    \item Create and navigate to the new project:
    \begin{verbatim}
    npx react-native init SimpleTodoApp
    cd SimpleTodoApp
    \end{verbatim}
    \begin{figure}[H]
        \centering
        \includegraphics[width=0.5\textwidth]{Settingup the project.jpg}
        \caption{Setting up the To-Do List Project}
    \end{figure}
    
    \item Open the project in Visual Studio Code:
    \begin{verbatim}
    code .
    \end{verbatim}
\end{enumerate}

\subsection{Step 2: Create the Basic To-Do List Structure}
Replace the content of \texttt{App.js} with the following code:
\begin{figure}[H]
    \centering
    \includegraphics[width=0.5\textwidth]{initial on emu.jpg}
    \caption{Initial View of To-Do List App on Android Emulator}
\end{figure}
\begin{figure}[H]
    \centering
    \includegraphics[width=0.5\textwidth]{initial on pd.jpg}
    \caption{Initial View of To-Do List App on Physical Device}
\end{figure}

\subsection{Explanation of the Code}
\begin{itemize}
    \item \textbf{State Management}
    \begin{itemize}
        \item The \texttt{useState} hook is used to manage the state of the input field (task) and the list of tasks (tasks).
        \item When a new task is added, it updates the tasks array, and the input field is cleared.
    \end{itemize}
    
    \item \textbf{Adding a Task}
    \begin{itemize}
        \item The \texttt{addTask} function checks if the input is not empty.
        \item It adds a new task with a unique ID (using the current timestamp) to the tasks array.
        \item The input field is then reset to an empty string.
    \end{itemize}
    
    \item \textbf{Deleting a Task}
    \begin{itemize}
        \item The \texttt{deleteTask} function filters out the task with the specified ID from the tasks array.
        \item This updates the state and re-renders the list without the deleted task.
    \end{itemize}
    
    \item \textbf{Rendering the List}
    \begin{itemize}
        \item The \texttt{FlatList} component efficiently renders the list of tasks.
        \item Each item in the list displays the task text and a delete button.
    \end{itemize}
\end{itemize}

\subsection{Step 4: Running the App}
\begin{enumerate}
    \item In your terminal, run:
    \begin{verbatim}
    npx react-native run-android
    \end{verbatim}
    or
    \begin{verbatim}
    npx react-native run-ios
    \end{verbatim}
    \item This compiles and runs your app on the selected platform.
    \begin{figure}[H]
        \centering
        \includegraphics[width=0.5\textwidth]{Running on emulator.jpg}
        \caption{Running the To-Do List App on Android Emulator}
    \end{figure}
    \begin{figure}[H]
        \centering
        \includegraphics[width=0.5\textwidth]{running on physical device.jpg}
        \caption{Running the To-Do List App on Physical Device}
    \end{figure}
\end{enumerate}

\subsection{Submission (Total 60 Points)}
Provide detailed answers to the following questions, including any necessary screenshots:

\textbf{Extending Functionality (60 Points)}
\begin{itemize}
    \item \textbf{Mark Tasks as Complete (15 Points)}
    \begin{itemize}
        \item Add a toggle function that allows users to mark tasks as completed.
        \item Style completed tasks differently, such as displaying strikethrough text or changing the text color.
        \begin{figure}[H]
            \centering
            \includegraphics[width=0.5\textwidth]{completed on emu.jpg}
            \caption{Task Marked as Complete on Emulator}
        \end{figure}
        \begin{figure}[H]
            \centering
            \includegraphics[width=0.5\textwidth]{completed on PD.jpg}
            \caption{Task Marked as Complete on Physical Device}
        \end{figure}
        \item Explain how you updated the state to reflect the completion status of tasks.
        \item In the state array for tasks, I added to each task object a property called "completed" to keep the state for completion status. This will enable that, upon clicking on the "mark as complete" button, the fired function is toggleCompletion; finding the task by ID, and then toggling its property "completed" between true and false. Then, it updates the state through setTasks to ensure the UI will re-render with the correct completion status for each task.
    \end{itemize}
    
    \item \textbf{Persist Data Using AsyncStorage (15 Points)}
    \begin{itemize}
        \item Implement data persistence so that tasks are saved even after the app is closed.
        \item Use AsyncStorage to store and retrieve the tasks list.
        \begin{figure}[H]
            \centering
            \includegraphics[width=0.5\textwidth]{async.jpg}
            \caption{AsyncStorage Code Snippet}
        \end{figure}
    \end{itemize}
    
    \item \textbf{Edit Tasks (10 Points)}
    \begin{itemize}
        \item Allow users to tap on a task to edit its content.
        \item Implement an update function that modifies the task in the state array.
        \begin{figure}[H]
            \centering
            \includegraphics[width=0.5\textwidth]{updated task on emu.jpg}
            \caption{Editing Task in the To-Do List App on Emulator}
        \end{figure}
        \begin{figure}[H]
            \centering
            \includegraphics[width=0.5\textwidth]{Updating on pd.jpg}
            \caption{Editing Task in the To-Do List App on Physical Device}
        \end{figure}
        \item Explain how you managed the UI for editing tasks.
        \item To manage the UI for editing tasks, I have implemented an edit mode for each task. Upon clicking, it reveals an input field with a current value of a task that can be changed by a user. I used conditional rendering to either show an input box or task text, depending on if the task was in edit mode. Once the user is through with editing and confirms the change, the changed value is written back into the state of the task, and edit mode is turned off. This makes the transition between viewing and editing tasks absolutely seamless.
    \end{itemize}
    
    \item \textbf{Add Animations (10 Points)}
    \begin{itemize}
        \item Use the \texttt{Animated} API from React Native to add visual effects when adding or deleting tasks.
        \item Describe the animations you implemented and how they enhance user experience.
             \item I have used React Native's Animated API for fade-in and slide-out animation. When a new task gets added, it should fade in-fashion so that the user gets to feel the transition being smooth and natural into the list. Then, when the task is deleted, this is told visually by the slide-out that one item is removed. These animations, in turn, provide improved feedback to the users, make the interactions more engaging, and bestow a fine, polished, modern touch upon the application. This does not only make the app more fascinating to use but also keeps the user better informed about the state changes happening within the app.
        \begin{figure}[H]
            \centering
            \includegraphics[width=0.5\textwidth]{animatoin.jpg}
            \caption{Code Snippet for Adding Animations Using Animated API}
        \end{figure}
    \end{itemize}
\end{itemize}

\section{Conclusion}
This report provides an overview of setting up a development environment for React Native, configuring an emulator, comparing development options, and building a simple To-Do List app with extended functionality. The setup allows for efficient app development, testing, and deployment on both emulators and physical devices.

\section{Acknowledgment of LLM Assistance}
During the course of setting up the development environment and fixing certain errors, I utilized a Language Learning Model (LLM) for assistance. Specifically, the LLM provided solutions for resolving JAVA\_HOME path verification issues and offered advice on handling the default \texttt{App.tsx} compilation problem.

\end{document}